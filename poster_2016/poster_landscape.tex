

\documentclass[landscape,final,a0paper,fontscale=0.285]{baposter}
    \renewcommand{\refname}{}
\usepackage{calc}
\usepackage{graphicx}
\usepackage{epstopdf}
\usepackage{amsmath}
\usepackage{amssymb}
\usepackage{relsize}
\usepackage{multirow}
\usepackage{rotating}
\usepackage{bm}
\usepackage{url}
\usepackage{cite}
\usepackage{graphicx}
\usepackage{multicol}

%\usepackage{times}
%\usepackage{helvet}
%\usepackage{bookman}
\usepackage{palatino}



\newcommand{\captionfont}{\footnotesize}

\graphicspath{{images/}{../images/}}
\usetikzlibrary{calc}

\newcommand{\SET}[1]  {\ensuremath{\mathcal{#1}}}
\newcommand{\MAT}[1]  {\ensuremath{\boldsymbol{#1}}}
\newcommand{\VEC}[1]  {\ensuremath{\boldsymbol{#1}}}
\newcommand{\Video}{\SET{V}}
\newcommand{\video}{\VEC{f}}
\newcommand{\track}{x}
\newcommand{\Track}{\SET T}
\newcommand{\LMs}{\SET L}
\newcommand{\lm}{l}
\newcommand{\PosE}{\SET P}
\newcommand{\posE}{\VEC p}
\newcommand{\negE}{\VEC n}
\newcommand{\NegE}{\SET N}
\newcommand{\Occluded}{\SET O}
\newcommand{\occluded}{o}


%%%%%%%%%%%%%%%%%%%%%%%%%%%%%%%%%%%%%%%%%%%%%%%%%%%%%%%%%%%%%%%%%%%%%%%%%%%%%%%%
%%%% Some math symbols used in the text
%%%%%%%%%%%%%%%%%%%%%%%%%%%%%%%%%%%%%%%%%%%%%%%%%%%%%%%%%%%%%%%%%%%%%%%%%%%%%%%%

%%%%%%%%%%%%%%%%%%%%%%%%%%%%%%%%%%%%%%%%%%%%%%%%%%%%%%%%%%%%%%%%%%%%%%%%%%%%%%%%
% Multicol Settings/
%%%%%%%%%%%%%%%%%%%%%%%%%%%%%%%%%%%%%%%%%%%%%%%%%%%%%%%%%%%%%%%%%%%%%%%%%%%%%%%%
\setlength{\columnsep}{1.5em}
\setlength{\columnseprule}{0mm}

%%%%%%%%%%%%%%%%%%%%%%%%%%%%%%%%%%%%%%%%%%%%%%%%%%%%%%%%%%%%%%%%%%%%%%%%%%%%%%%%
% Save space in lists. Use this after the opening of the list
%%%%%%%%%%%%%%%%%%%%%%%%%%%%%%%%%%%%%%%%%%%%%%%%%%%%%%%%%%%%%%%%%%%%%%%%%%%%%%%%
\newcommand{\compresslist}{%
\setlength{\itemsep}{1pt}%
\setlength{\parskip}{0pt}%
\setlength{\parsep}{0pt}%
}

%%%%%%%%%%%%%%%%%%%%%%%%%%%%%%%%%%%%%%%%%%%%%%%%%%%%%%%%%%%%%%%%%%%%%%%%%%%%%%
%%% Begin of Document
%%%%%%%%%%%%%%%%%%%%%%%%%%%%%%%%%%%%%%%%%%%%%%%%%%%%%%%%%%%%%%%%%%%%%%%%%%%%%%

\begin{document}

%%%%%%%%%%%%%%%%%%%%%%%%%%%%%%%%%%%%%%%%%%%%%%%%%%%%%%%%%%%%%%%%%%%%%%%%%%%%%%
%%% Here starts the poster
%%%---------------------------------------------------------------------------
%%% Format it to your taste with the options
%%%%%%%%%%%%%%%%%%%%%%%%%%%%%%%%%%%%%%%%%%%%%%%%%%%%%%%%%%%%%%%%%%%%%%%%%%%%%%
% Define some colors

%\definecolor{lightblue}{cmyk}{0.83,0.24,0,0.12}
\definecolor{lightblue}{rgb}{0.145,0.6666,1}

% Draw a video
\newlength{\FSZ}
\newcommand{\drawvideo}[3]{% [0 0.25 0.5 0.75 1 1.25 1.5]
   \noindent\pgfmathsetlength{\FSZ}{\linewidth/#2}
   \begin{tikzpicture}[outer sep=0pt,inner sep=0pt,x=\FSZ,y=\FSZ]
   \draw[color=lightblue!50!black] (0,0) node[outer sep=0pt,inner sep=0pt,text width=\linewidth,minimum height=0] (video) {\noindent#3};
   \path [fill=lightblue!50!black,line width=0pt] 
     (video.north west) rectangle ([yshift=\FSZ] video.north east) 
    \foreach \x in {1,2,...,#2} {
      {[rounded corners=0.6] ($(video.north west)+(-0.7,0.8)+(\x,0)$) rectangle +(0.4,-0.6)}
    }
;
   \path [fill=lightblue!50!black,line width=0pt] 
     ([yshift=-1\FSZ] video.south west) rectangle (video.south east) 
    \foreach \x in {1,2,...,#2} {
      {[rounded corners=0.6] ($(video.south west)+(-0.7,-0.2)+(\x,0)$) rectangle +(0.4,-0.6)}
    }
;
   \foreach \x in {1,...,#1} {
     \draw[color=lightblue!50!black] ([xshift=\x\linewidth/#1] video.north west) -- ([xshift=\x\linewidth/#1] video.south west);
   }
   \foreach \x in {0,#1} {
     \draw[color=lightblue!50!black] ([xshift=\x\linewidth/#1,yshift=1\FSZ] video.north west) -- ([xshift=\x\linewidth/#1,yshift=-1\FSZ] video.south west);
   }
   \end{tikzpicture}
}

\hyphenation{resolution occlusions}
%%
\begin{poster}%
  % Poster Options
  {
  % Show grid to help with alignment
  grid=false,
  % Column spacing
  colspacing=1em,
  % Color style
  bgColorOne=white,
  bgColorTwo=white,
  borderColor=lightblue,
  headerColorOne=black,
  headerColorTwo=lightblue,
  headerFontColor=white,
  boxColorOne=white,
  boxColorTwo=lightblue,
  % Format of textbox
  textborder=roundedleft,
  % Format of text header
  eyecatcher=true,
  headerborder=closed,
  headerheight=0.1\textheight,
%  textfont=\sc, An example of changing the text font
  headershape=roundedright,
  headershade=shadelr,
  headerfont=\Large\bf\textsc, %Sans Serif
  textfont={\setlength{\parindent}{1.5em}},
  boxshade=plain,
%  background=shade-tb,
  background=plain,
  linewidth=2pt
  }
  % Eye Catcher
  {\includegraphics[height=5em]{img/lgpsealm.eps}}
  % 
  {\bf\textsc{Web Interface for daylighting design and feedback }\vspace{0.5em}}
  % Authors
  {\textsc{\{ Max Espinoza, Rebecca Nordhauser, Joshua Nasman, and Barbara Cutler \}}}
  % University logo
  {% The makebox allows the  to flow into the logo, this is a hack because of the L shaped logo.
    \includegraphics[height=5.0em]{img/logo_cs_design_1.png}
  }

%%%%%%%%%%%%%%%%%%%%%%%%%%%%%%%%%%%%%%%%%%%%%%%%%%%%%%%%%%%%%%%%%%%%%%%%%%%%%%
%%% Now define the boxes that make up the poster
%%%---------------------------------------------------------------------------
%%% Each box has a name and can be placed absolutely or relatively.
%%% The only inconvenience is that you can only specify a relative position 
%%% towards an already declared box. So if you have a box attached to the 
%%% bottom, one to the top and a third one which should be in between, you 
%%% have to specify the top and bottom boxes before you specify the middle 
%%% box.
%%%%%%%%%%%%%%%%%%%%%%%%%%%%%%%%%%%%%%%%%%%%%%%%%%%%%%%%%%%%%%%%%%%%%%%%%%%%%%
    %
    % A coloured circle useful as a bullet with an adjustably strong filling
    \newcommand{\colouredcircle}{%
      \tikz{\useasboundingbox (-0.2em,-0.32em) rectangle(0.2em,0.32em); \draw[draw=black,fill=lightblue,line width=0.03em] (0,0) circle(0.18em);}}

%%%%%%%%%%%%%%%%%%%%%%%%%%%%%%%%%%%%%%%%%%%%%%%%%%%%%%%%%%%%%%%%%%%%%%%%%%%%%%
  \headerbox{Motivation}{name=problem,column=0,row=0}{
%%%%%%%%%%%%%%%%%%%%%%%%%%%%%%%%%%%%%%%%%%%%%%%%%%%%%%%%%%%%%%%%%%%%%%%%%%%%%%


  %Images for this section
  %{\includegraphics[height=10em]{img/office_color.jpg}}
  {\includegraphics[height=10.25em]{img/oasis_1.png}}

  {\includegraphics[height=13em]{img/aag_figure_3.png}}


  {\includegraphics[height=10.25em]{img/oasis_2.png}} 
  %{\includegraphics[height=11.25em]{img/oasis_2.png}}

  %text for this section
Previous research on the development of a spatially augmented tangible user interface for architectural daylighting simulation and design offered architects a novel approach in collaborative room design. \cite{NasmanEvaluation, ShengYYC09 }

User studies found the tools very helpful for visualizing lighting simulations and offered visual feedback to make better use of natural daylighting. Evaluation from this study lead to extending the interface to include avatars, illumination visualizations, and additional window models.\cite{NasmanEvaluation}

In order to collect more data we implemented an online interface which deliverers a set of capabilities similar to the physical system. Users are able to sketch designs which are translated into 3D models. Daylight simulated on these models relates illumination values to real world measurements.

 }%end section


%%%%%%%%%%%%%%%%%%%%%%%%%%%%%%%%%%%%%%%%%%%%%%%%%%%%%%%%%%%%%%%%%%%%%%%%%%%%%%
\headerbox{User Interface and Interactive Feedback}{name=userinterface,span = 2,column=2,,row=0}{
  %%%%%%%%%%%%%%%%%%%%%%%%%%%%%%%%%%%%%%%%%%%%%%%%%%%%%%%%%%%%%%%%%%%%%%%%%%%%%%
  {\includegraphics[width=42em]{img/viewer.jpg}}
  
  \begin{multicols}{2}
  	\textbf{User Interface:} The interface offers a variety of tools to design and create floor layouts including wall selection, window placement, furniture assignment, and compass orientation.
  	
  	\textbf{Interactive Feedback:} After a design is created, selecting the next tab in the navigation bar will save the sketch and render a 3D model of the floor plan (A). From the model, the user can create a rendering of the room at any given time of day (B). The resulting rendering is interactive such that users can navigate around the model to view lighting from different view points. This view provides feedback which the user can then use to alter their design and optimize usage of natural daylighting. \end{multicols}
  \vspace{-0.6em}
   

}

%%%%%%%%%%%%%%%%%%%%%%%%%%%%%%%%%%%%%%%%%%%%%%%%%%%%%%%%%%%%%%%%%%%%%%%%%%%%%%
  \headerbox{Contributions}{name=contribution,column=1,row=0}{
%%%%%%%%%%%%%%%%%%%%%%%%%%%%%%%%%%%%%%%%%%%%%%%%%%%%%%%%%%%%%%%%%%%%%%%%%%%%%%
  
   Currently a working prototype of the application offers functionality similar to the physical system.
   We used WebGL and the Raphael graphics libraries to build both the user interface and 3D visualization. Our online application implements the following features.
   \begin{enumerate}\compresslist
   \item A simple drag and drop interface to design models with wall primitives
   %\item Avatar primitives that render geometry to view a POV scene
   \item Save and load user created layouts
   \item Interactive feedback that allows navigation through user generated spaces with simulated daylight
   \end{enumerate}
   
\begin{center}{\includegraphics[width=15em]{img/new_pipeline.png}}\end{center}
The previous pipe line used a camera and image detection to create the primitive positions\cite{NasmanEvaluation, ShengYYC09 }, however with our online application we bypass image recognition and directly find the wall primitives and save those positions onto a database for later study. Those primitives go into the graphics pipeline and we get a triangle mesh model and rendered images with lighting textures, which are displayed on the screen as a 3D model. 
\vspace{1em}
    
      
   

  }

%%%%%%%%%%%%%%%%%%%%%%%%%%%%%%%%%%%%%%%%%%%%%%%%%%%%%%%%%%%%%%%%%%%%%%%%%%%%%%
\headerbox{Current Research}{name=future,column=2,row=1,below=userinterface,span = 2,bottomaligned=problem}{

\begin{multicols}{2}
	
	\textbf{Pilot Study}: We launched OASIS to a small number of willing participants. There are a number of optional questions that we hoped the participants would answer. The participants were asked to navigate the site and create a few models. We created a short video to help participants navigate around the system. Some of the participants reported tiny bugs through our bug submission form. Almost all the bugs have since been reviewed and patched. Additionally the feedback from participants both experienced and inexperienced in daylighting monitoring have provided useful in setting goals for future improvements.
	

	 \begin{center}\includegraphics[height=6em]{img/videoqr.jpg}	
	  \end{center}	
	\textbf{Viewer}:  The designs created using OASIS belong to the user, but we have added the option for the user to share designs with anyone. By putting the models in an encapsulated viewer, the only action the user needs to do in order to share their design is provide the unique share link associated with a model. We plan to showcase popular models that people wish to share in a gallery. 
	
	\begin{center}\includegraphics[height=10em]{img/nab.png}	
	\end{center}	
	
	
	
	   
 \end{multicols}
}
\headerbox{References}{name=credit, column = 1, row = 1, below = contribution, span = 1,bottomaligned=problem}{
{\tiny

\bibliography{cites}{}
\clearpage
\bibliographystyle{plain}

\par}
}

%%%%%%%%%%%%%%%%%%%%%%%%%%%%%%%%%%%%%%%%%%%%%%%%%%%%%%%%%%%%%%%%%%%%%%%%%%%%%
%%%%%%%%%%%%%%%%%%%%%%%%%%%%%%%%%%%%%%%%%%%%%%%%%%%%%%%%%%%%%%%%%%%%%%%%%%%%%%

\end{poster}
\end{document}












